\documentclass[12pt]{article}
\usepackage{amsmath}
\usepackage{amsthm}
\usepackage{newtx}
\usepackage[margin=1.25in]{geometry}

\DeclareMathOperator{\val}{val}

\newtheorem{conj}{Conjecture}
\newtheorem{prop}[conj]{Proposition}

\begin{document}

\begin{center}
  \textsc{\large
    Paul's Poset Poser
  }
\end{center}

\begin{itemize}
\item For a positive integer $b$ let $W_b$ be the set of all
  length-$b$ binary words.

\item
  For $w \in W_b$, let $\val w$ be the integer represented in binary
  by $w$; that is
  \[
  \val w = \sum_{i=1}^b w_i 2^{b-i} .
  \]
  For example, $\val (1,1,0,1) = 1101_{\textsc{two}} = 13$.

\item For $v,w \in W_b$, we have $v<w$ exactly when $\val v < \val w$.

\item For $w \in W_b$, let $\rho(w)$ be the \emph{reversal} of
  $w$. That is $$\rho(d_1,d_2,\ldots,d_b) = (d_b,d_{b-1},\ldots,d_1).$$

\item A \emph{chain} is a sequence of words $w_1,w_2,\ldots,w_n\in
  W_b$ such that $w_1 < w_2 < \cdots <w_n$ and $\rho(w_1) < \rho(w_2) <
  \cdots < \rho(w_n)$. 

\item An \emph{antichain} is a sequence of words $w_1,w_2,\ldots,w_n\in
  W_b$ such that $w_1 < w_2 < \cdots < w_n$ and $\rho(w_1) > \rho(w_2) >
  \cdots > \rho(w_n)$. 

\item Let $c(b)$ be the size of a maximum chain in $W_b$ and $a(b)$ be
  the size of a maximum antichain in $W_b$. 

\item If $v$ and $w$ are binary words, $vw$ is their concatenation. 
\end{itemize}

\begin{conj}
  For a positive integer $b$,
  \[
  c(b) = 
  \begin{cases}
    3\cdot 2^{b/2-1} & \text{if $b$ is even and} \\
    2^{(b+1)/2} & \text{if $b$ is odd}.
  \end{cases}
  \]
\end{conj}

\begin{conj}
  For a positive integer $b$, $a(b) = c(b)-1$. 
\end{conj}


\newpage

\begin{prop} \label{prop:c-lower}
    For a positive integer $b$,
  \[
  c(b) \ge
  \begin{cases}
    3\cdot 2^{b/2-1} & \text{if $b$ is even and} \\
    2^{(b+1)/2} & \text{if $b$ is odd}.
  \end{cases}
  \]
\end{prop}

\begin{proof}[Sketch of Proof]
  Observe that if $w_1,w_2,\ldots,w_n$ is a chain in $W_b$, then 
  $$0w_10, 0w_20, \ldots, 0w_n0, 1w_11, 1w_21, \ldots 1w_n1$$ is a chain
  in $W_{b+2}$ that is twice as long. 

  We see that $c(1)=2$ because $0,1$ is a chain and that $c(2)=3$
  because $00,01,11$ is a chain (and easy to check we cannot make a
  longer one). 

  The result now follows by induction on $b$. 
\end{proof}


\begin{proof}[Alternative sketch proof for odd $b$]
  The palindromes of $W_b$ form a chain. When $b$ is odd, the number
  of palindromes is $2^{(b+1)/2}$. 
\end{proof}

\end{document}
