\documentclass[12pt]{article}
\usepackage{amsmath}
\usepackage{amsthm}
\usepackage{newtx}
\usepackage[margin=1.5in]{geometry}

\DeclareMathOperator{\val}{val}

\newtheorem{conj}{Conjecture}
\newtheorem{prop}[conj]{Proposition}

\begin{document}

\begin{center}
  \textsc{\large
    Paul's Poset Poser
  }
\end{center}

\begin{itemize}
\item For a positive integer $b$ let $W_b$ be the set of all
  length-$b$ binary words.

\item
  For $w \in W_b$, let $\val w$ be the integer represented in binary
  by $w$; that is
  \[
  \val w = \sum_{i=1}^b w_i 2^{b-i} .
  \]
  For example, $\val (1,1,0,1) = 1101_{\textsc{two}} = 13$.

\item For $w \in W_b$, let $\rho(w)$ be the \emph{reversal} of
  $w$. That is $$\rho(d_1,d_2,\ldots,d_b) =
  (d_b,d_{b-1},\ldots,d_1).$$

\item For words $v,w\in W_b$, define $v \prec w$ provided $\val v <
  \val w$ and $\val \rho(v) < \val \rho(w)$. Note that $(W_b,\prec)$
  is a poset. 

\item A \emph{chain} is a subset of $W_b$ containing
  pairwise comparable words and an \emph{antichain} is a subset of
  $W_b$ containing pairwise incomparable words. 



\item Let $c(b)$ be the size of a maximum chain in $W_b$ and $a(b)$ be
  the size of a maximum antichain in $W_b$. 

\item If $v$ and $w$ are binary words, $vw$ is their concatenation. 
\end{itemize}

\begin{conj}
  For a positive integer $b$,
  \[
  c(b) = 
  \begin{cases}
    3\cdot 2^{b/2-1} & \text{if $b$ is even and} \\
    2^{(b+1)/2} & \text{if $b$ is odd}.
  \end{cases}
  \]
\end{conj}

\begin{conj}
  For a positive integer $b$, $a(b) = c(b)-1$. 
\end{conj}


\newpage

\begin{prop}[Chain lower bound]\label{prop:c-lower}
    For a positive integer $b$,
  \[
  c(b) \ge
  \begin{cases}
    3\cdot 2^{b/2-1} & \text{if $b$ is even and} \\
    2^{(b+1)/2} & \text{if $b$ is odd}.
  \end{cases}
  \]
\end{prop}

\begin{proof}[Proof sketch]
  Observe that if $w_1\prec w_2 \prec \cdots \prec w_n$ is a chain in $W_b$, then 
  $$
  0w_10 \prec 0w_20 \prec \cdots \prec 0w_n0 \prec 1w_11\prec
  1w_21\prec \cdots \prec 1w_n1
  $$ is a chain
  in $W_{b+2}$ that is twice as long.

  We see that $c(1)=2$ because $0\prec1$ is a chain in $W_1$ and that
  $c(2)=3$ because $00\prec01\prec11$ is a chain in $W_2$ (and easy to
  check we cannot make a longer one).

  The result now follows by induction on $b$. 
\end{proof}


\begin{proof}[Alternative proof sketch for odd $b$]
  The palindromes of $W_b$ form a chain. When $b$ is odd, the number
  of palindromes is $2^{(b+1)/2}$. This argument doesn't work for
  even $b$.
\end{proof}

\end{document}
