\documentclass[12pt]{article}
\usepackage{amsmath}
\usepackage{amsthm}
\usepackage{newtx}
\usepackage[margin=1.5in]{geometry}

\DeclareMathOperator{\val}{val}
\newcommand{\two}{\textsc{two}}

\newcommand{\barv}{\bar{v}}
\newcommand{\barw}{\bar{w}}

\newcommand{\rhov}{\rho v}
\newcommand{\rhow}{\rho w}

\newcommand{\rhobarv}{\rho \barv}
\newcommand{\rhobarw}{\rho \barw}

\newtheorem{conj}{Conjecture}
\newtheorem{corr}[conj]{Corrolary}
\newtheorem{prop}[conj]{Proposition}
\newtheorem{thm}[conj]{Theorem}

\newcommand{\newthought}{%
  \smallskip\centerline{\rule{2in}{0.5pt}}\smallskip}

\begin{document}

\begin{center}
  \textsc{\large
    Evan's Theorem
  }
\end{center}

Let $v$ be a $b$-bit word. Treat $v$ as a binary number to get its
value. For example, $\val (1,1,0,0) = 1100_\two = 12$. We have $v\prec
w$ exactly when $\val v < \val w$ and $\val \rhov < \val \rhow$.


\begin{itemize}
\item A \emph{type I} transformation replaces a $0$ in $v$ with a $1$.

  Example: $001\underline0 101 \to 001 \underline1 101$. 

\item A \emph{type H} transformation replaces a substring in $v$ of
  the form\footnote{The substring begins and ends with a $0$ and has
    one or more consecutive $1$s in between.}  $011\dots10$ with its
  complement, i.e., $100\ldots01$.

  Example: $0\underline{0110}01 \to 0 \underline{1001} 01$.
\end{itemize}


\begin{prop}
  Suppose $v,w\in W_b$ and $w$ is derived from $v$ by an I
  transformation, than $v \prec w$.
\end{prop}


\begin{proof}
  Let $v=(x,0,y)$ and $w=(x,1,y)$. Then $\val w = \val v + 2^k > \val
  v$ for some $k$. Likewise, $\val\rho w = \val\rho v + 2^j > \val\rho
  v$ for some $j$. Therefore $v \prec w$.
\end{proof}

\begin{prop}
  Suppose $v,w\in W_b$ and $w$ is derived from $v$ by an H
  transformation, than $v \prec w$.
\end{prop}

\begin{proof}
  Let $v = (x,01\ldots10,y)$ and $w=(x,10\ldots01,y)$. Note that
  $10\ldots01_\textsc{two} >01\ldots10_\two$ and therefore $\val w
  >\val v$. Likewise, $\val \rho w > \val \rho v$. Therefore $v \prec
  w$.  
\end{proof}


\begin{corr}
  Let $v,w \in W_b$. If $w$ is obtained from $v$ by a series of I and
  H transformations, then $v\preceq w$. \qed
\end{corr}

The interesting part is the converse.


\begin{thm}
  Let $v,w \in W_b$ with $v \prec w$. Then $w$ is obtained from $v$
  by a sequence of I and H transformations. 
\end{thm}

\begin{proof}
  The proof is by strong induction on $b$. It is easy to verify for
  $W_1$, $W_2$, and $W_3$. Suppose the result has been shown for all
  $W_i$ with $i<b$. 

  Let $v,w \in W_b$. We consider the first and last bits of both
  words, and break into cases. All told there are 16 possibilities:
  Four possibilities for $v$: $(0,\barv,0)$, $(0,\barv,1)$,
  $(1,\barv,0)$, and $(1,\barv,1)$, and likewise for $w$.


  Cases as follows:

  \begin{enumerate}
  \item $v=(0,\barv,0)$ and $w=(0,\barw,0)$: Induction on
    $\barv\prec\barw$. 

  \item $v=(0,\barv,0)$ and $w=(0,\barw,1)$: Induction on
    $(\barv,0)\prec(\barw,1)$. 


  \item $v=(0,\barv,0)$ and $w=(1,\barw,0)$: Induction on
    $(0,\barv)\prec(1,\barw)$. 


  \item $v=(0,\barv,0)$ and $w=(1,\barw,1)$: Use I transformations to
    go from $v=(0,\barv,0)$ to $(0,11\ldots1,0)$, then one H
    transformation to $(1,00\ldots0,1)$, and then I transformations
    to $w=(1,\barw,1)$.

    \newthought
    
  \item $v=(0,\barv,1)$ and $w=(0,\barw,0)$: Contradiction: $\val\rho v >
    \val\rho w$.


  \item $v=(0,\barv,1)$ and $w=(0,\barw,1)$: Induction on
    $(\barv,1)\prec(\barw,1)$. 


  \item $v=(0,\barv,1)$ and $w=(1,\barw,0)$: Contradiction:
    $\val\rho v > \val\rho w$. 


  \item $v=(0,\barv,1)$ and $w=(1,\barw,1)$: Induction on
    $(0,\barv)\prec(1,\barw)$. 

    
    \newthought

  \item $v=(1,\barv,0)$ and $w=(0,\barw,0)$: Contradiction: $\val v >
    \val w$. 

  \item $v=(1,\barv,0)$ and $w=(0,\barw,1)$:  Contradiction: $\val v >
    \val w$. 

  \item $v=(1,\barv,0)$ and $w=(1,\barw,0)$: Induction on
    $(1,\barv)\prec(1,\barw)$. 

  \item $v=(1,\barv,0)$ and $w=(1,\barw,1)$: Induction on
    $(1,\barv)\prec(1,\barw)$.

    \newthought
    
  \item $v=(1,\barv,1)$ and $w=(0,\barw,0)$: Contradiction: $\val v >
    \val w$.


  \item $v=(1,\barv,1)$ and $w=(0,\barw,1)$: Contradiction: $\val v >
    \val w$.


  \item $v=(1,\barv,1)$ and $w=(1,\barw,0)$: Contradiction: $\val \rho
    v > \val \rho w$.


  \item $v=(1,\barv,1)$ and $w=(1,\barw,1)$: Induction on
    $\barv\prec\barw$. \qedhere





  \end{enumerate}
\end{proof}



Recall that, in a poset, $v$ is \emph{covered by} $w$ provided $v
\prec w$ and there is no element $x$ with $v \prec x \prec w$. 


\begin{corr}
  For $v,w \in W_b$, $v$ is covered by $w$ if and only if $w$ is
  obtained from $v$ by a type I or type H transformation.\qed
\end{corr}

In particular, if $w_1 \prec w_2 \prec \cdots \prec w_t$ is a maximum
chain in $W_b$, then each $w_{i+1}$ is obtained from $w_i$ by a type~I
or a type~H transformation. 


\end{document}

Conjecture is well-supported by computer evidence.

Some chain examples:

\small

\begin{verbatim}
julia> show_transformations(palindromes(7))
0000000 → 0001000       Type I 
0001000 → 0010100       Type H
0010100 → 0011100       Type I 
0011100 → 0100010       Type H
0100010 → 0101010       Type I 
0101010 → 0110110       Type H
0110110 → 0111110       Type I 
0111110 → 1000001       Type H
1000001 → 1001001       Type I 
1001001 → 1010101       Type H
1010101 → 1011101       Type I 
1011101 → 1100011       Type H
1100011 → 1101011       Type I 
1101011 → 1110111       Type H
1110111 → 1111111       Type I 

julia> show_transformations(max_chain(7))
0000000 → 0010000       Type I 
0010000 → 0011000       Type I 
0011000 → 0100100       Type H
0100100 → 0110100       Type I 
0110100 → 0111100       Type I 
0111100 → 1000010       Type H
1000010 → 1010010       Type I 
1010010 → 1011010       Type I 
1011010 → 1011110       Type I 
1011110 → 1100001       Type H
1100001 → 1101001       Type I 
1101001 → 1101101       Type I 
1101101 → 1110011       Type H
1110011 → 1111011       Type I 
1111011 → 1111111       Type I 

julia> show_transformations(recursive_chain(7))
0000000 → 0001000       Type I 
0001000 → 0010100       Type H
0010100 → 0011100       Type I 
0011100 → 0100010       Type H
0100010 → 0101010       Type I 
0101010 → 0110110       Type H
0110110 → 0111110       Type I 
0111110 → 1000001       Type H
1000001 → 1001001       Type I 
1001001 → 1010101       Type H
1010101 → 1011101       Type I 
1011101 → 1100011       Type H
1100011 → 1101011       Type I 
1101011 → 1110111       Type H
1110111 → 1111111       Type I
\end{verbatim}


\end{document}
